\chapter{Проектный раздел}

\section{Информационное обеспечение ИС}

\subsection{Входные и выходные данные}

\lipsum[1][1-3]

\subsection{Информационная модель процессов}

\lipsum[2][1-3]

\subsection{Путь пользователя}

Путь пользователя (User Flow) описывает последовательность действий
при работе с консольным приложением для конвертации данных,
как показано на рисунке~\ref{fig:user:flow}.
Процесс начинается с запуска программы через командную строку.

\begin{image}
   %\includegrph{user_flow.drawio.png}  % file in ./img
   \caption{Путь пользователя}
   \label{fig:user:flow}
\end{image}

\section{Архитектура программного решения}

\subsection{Архитектурный паттерн проектирования}

\subsection{Архитектурная диаграмма}

\subsubsection{Диаграмма контекста системы}

\subsubsection{Диаграмма компонентов}

\subsection{Масштабируемость системы}

\subsection{Описание инструментов для компонентов архитектуры}

\subsubsection{Обработка пользовательского ввода}

\subsubsection{Парсинг}

\subsubsection{Предобработка}

\subsubsection{Компонент конвертации}

\subsubsection{Компонент генерации отчёта}

\section{Математическое обеспечение}

\subsection{Алгоритмы для построения ориентированных ациклических графов}
\label{section:algs:dag}

Ориентированные ациклические графы (DAG) являются ключевым инструментом
для моделирования зависимостей между элементами в DRC-скриптах.
Каждый элемент скрипта представляется вершиной графа,
а зависимости между ними~--- ориентированными рёбрами.
Такое представление позволяет визуализировать
и анализировать порядок выполнения операций, исключая конфликты и циклы.

Формально DAG определяется как упорядоченная пара~\eqref{eq:dag}.

\begin{equation}\label{eq:dag}
G = (V, E)
\end{equation}

где \( G \)~--- ориентированные ациклические графы,
\( V \)~--- множество вершин,
представляющих элементы DRC скрипта,
\( E \)~--- множество рёбер,
где каждое ребро \( (u, v) \) из вершины \( u \) в вершину \( v \)
обозначает зависимость между двумя элементами.

Направленность рёбер гарантирует однозначное определение зависимостей 
(например, \( u \to v \) означает, что \( v \) выполняется после \( u \)).
Ацикличность графа исключает возможность зацикливания операций,
что критично для корректного выполнения скрипт.
Например, цикл вида \( u \to v \to w \to u \)
сделал бы выполнение логически невозможным.

\subsection{Алгоритмы для топологической сортировки DAG}
\label{section:algs:topology-sort-dag}

Топологическая сортировка ориентированного ациклического графа (DAG)~---
это процесс упорядочивания вершин,
при котором для каждого ребра \( (u, v) \)
вершина \( u \) предшествует вершине \( v \).
Этот порядок гарантирует соблюдение всех зависимостей в DRC-скриптах.

Для графа \( G = (V, E) \)
топологическая сортировка представляет
собой биективное отображение~\eqref{eq:topology:tau}.

\begin{equation}\label{eq:topology:tau}
	\tau: V \to \{1, 2, \ldots, |V|\}
\end{equation}

где для любого ребра \( (u, v) \in E \)
выполняется условие \( \tau(u) < \tau(v) \).

Это условие обеспечивает корректную последовательность операций,
исключая обработку зависимого элемента \( v \) до завершения \( u \).

Есть два ключевых алгоритма реализации топологической сортировки:

\begin{enumerate}
    \item Алгоритм Кана, гдe
		порядок вычисляется, используя степени входа вершин.
		При этом выполняются следующие шаги:
		\begin{itemize}
			\item инициализация очереди вершин с нулевой входящей степенью;
			\item последовательное удаление вершин из очереди,
				уменьшение степеней их потомков;
			\item добавление вершин с обнулёнными степенями в результат.
		\end{itemize}
	\item DFS-подход (Тарьян), который
		основан на поиске в глубину с постобработкой.
		То есть вершины добавляются
		в результат после посещения всех их потомков.
\end{enumerate}

У топологической сортировки можно выделить следующие свойства:

\begin{itemize}
    \item неединственность.
		DAG может иметь несколько допустимых порядков;
    \item ацикличность.
		Алгоритмы автоматически обнаруживают циклы,
		что делает их применимыми только для DAG.
\end{itemize}

При проверке правил проектирования микросхем топологическая
сортировка определяет порядок выполнения операций.
Например, проверка расстояний между слоями (элемент \( v \))
требует предварительного расчёта координат (элемент \( u \)),
что отражается порядком \( \tau(u) < \tau(v) \).

\section{Оценка ресурсов и сложности реализации}

