\chapter*{Введение}
\addcontentsline{toc}{chapter}{Введение}

\textit{Актуальность.}
\lipsum[1]

\underline{Цель работы:}
\lipsum[2][1-4]

\underline{Объект исследования:}
\lipsum[3][1]

\underline{Предмет исследования:}
\lipsum[4][1]

\underline{Значимость работы:}
Реализация данного проекта направлена на решение важной
задачи повышения эффективности процессов проектирования за счет автоматизации.
Это соответствует современным тенденциям в разработке
и обеспечивает конкурентные преимущества в отрасли.

Для достижения поставленной цели
необходимо решить следующие \underline{задачи}:

\begin{enumerate}
	\item характеристика предметной области;
	\item определение функциональных границ;
	\item анализ существующих разработок;
	\item формирование требований к ИС;
	\item проектирование информационного обеспечения,
		архитектуры, математических моделей и методик конвертации данных;
	\item обоснование выбора средств и технологий разработки;
	\item описание логики работы, модели данных и реализации ИС;
	\item демонстрация работоспособности;
	\item описание технологии работы с ИС;
	\item экономические параметры разработки и внедрения ИС.
\end{enumerate}

Данная выпускная квалификационная работа включает:
список используемых сокращений, введение, четыре главы, заключение,
список использованной литературы и приложения.

Введение предоставляет общий обзор выбранной темы, формулирует цель
и задачи исследования, а также обосновывает актуальность проблемы.
Этот раздел объясняет,
почему данная проблема важна для современной инженерной практики
и какие преимущества дает ее решение.

Первая глава посвящена исследовательскому разделу,
который охватывает контекст разработки,
включая объект исследования, целевую аудиторию
и рыночно-технологические условия.
Определяет функциональные границы ИС.
Обосновывает выбора исходного и целевого формата конвертации
и разработки собственного продукта из-за отсутствия адаптивных аналогов.
Формулирует функциональные, нефункциональные, пользовательские требования
и требования обеспечения к информационной системе,
а также требования к надежности и безопасности.
Завершается раздел выявлением, анализом и минимизацией
потенциальных угроз, способных повлиять на достижение целей разработки.

Второй раздел описывает проектные решения.
Определяется информационное обеспечение, описывающее входные и выходные данные,
информационную модель процессов,
а также сценарии взаимодействия пользователя с системой.
Разрабатывается архитектура программного решения,
включая описание компонентов, их масштабируемости и интеграции.
Описывается математическое обеспечение, охватывающее алгоритмы
построения ориентированного ациклического графа,
топологической сортировки (алгоритм Тарьяна).
Определяется методика работы программы для последующей реализации.
И проводится оценка ресурсов и сложности реализации.

Третий технологический раздел посвящен практической реализации ИС.
Раздел начинается с обоснования выбора средств разработки.
Затем рассматривается логика работы системы через дерево функций,
диаграмм прецедентов и последовательности,
а также модель данных, описывающая структуру хранения.
Для демонстрации реализации демонстрируется тестирование, охватывающее
модульные, интеграционные и нагрузочные тесты,
а также тесты пользовательского интерфейса.
Также корректность демонстрируется на примерах.
В завершении раздела описывается работа пользователей с разработаной ИС,
которая включает настройку окружения и запуск конвертации.

Четвертый раздел охватывает организационные и финансовые аспекты проекта,
включая планирование работ с определением участников
и организацию их деятельности.
Расчёт стоимости проведения работ включает 6 статей, такие как
сырьё и материалы, основная заработная плата,
дополнительная зарплата страховые взносы,
амортизация оборудования и прочие расходы.
Раздел завершается расчётом полной себестоимости работ.

Заключение подводит итоги исследования,
формулирует основные выводы и предлагает перспективы дальнейших исследований.
Здесь акцентируется внимание на значимости полученных результатов.

Список использованной литературы включает все источники,
использованные при выполнении выпускной квалификационной работы,
включая научные статьи, книги, публикации и интернет-ресурсы.

Приложения содержат дополнительные материалы, такие как техническое задание,
разработанное в соответствии с ГОСТ 34.602-2020~\cite{gost34_602},
акт о внедрении,
а также графический материал (презентация) выпускной квалификационной работы.

Написание выпускной квалификационной работы
руководствовалось следующими нормативными актами:

\begin{enumerate}
	\item <<О защите населения и территории
		от чрезвычайных ситуаций природного
		и техногенного характера>>
		от 21.12.1994 \No 68-ФЗ~\cite{russian_emergency_law_1994}.
	\item <<Об основах охраны здоровья граждан
		в Российской Федерации>>
		от 21.11.2011 \No 323-ФЗ~\cite{healthcare_law_2011}.
	\item <<О гражданской обороне>>
		от 12.02.1998 \No 28-ФЗ~\cite{civil_defense_law_1998}.
	\item Приказ Минздравсоцразвития РФ
		от 04.05.2012 \No 477н
		<<Об утверждении перечня состояний,
		при которых оказывается первая помощь,
		и перечня мероприятий
		по оказанию первой помощи>>~\cite{first_aid_order_2012}.
	\item Трудовой кодекс Российской Федерации
		от 30.12.2001 \No 197-ФЗ~\cite{labor_code_2001}.
	\item СанПиН 2.2.2/2.4.1340-03
		<<Гигиенические требования к персональным
		электронно-вычислительным машинам
		и организации работы>>~\cite{sanpin_2003}.
\end{enumerate}

