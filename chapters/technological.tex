\chapter{Технологический раздел}

\section{Выбор и обоснование средств разработки}

Для разработки программы используется
следующий набор инструментов и библиотек,
обеспечивающий гибкость, производительность и надежность:

\begin{enumerate}
	\item Python~--- основной язык программирования,
		выбранный за простоту интеграции с различными форматами данных,
		богатство библиотек и поддержку скриптовой автоматизации.
	\item NetworkX~--- инструмент для работы с графами.
		Применяется для моделирования зависимостей между правилами (DAG),
		оптимизации порядка их выполнения и предотвращения циклов.
	\item Pytest~--- фреймворк для модульного
		и интеграционного тестирования.
		Обеспечивает автоматическую проверку корректности конвертации
		и устойчивости кода.
	\item Git с GitHub/GitLab~--- система управления версиями
		и платформы для хранения кода.
		Позволяют организовать CI/CD-пайплайны,
		автоматизирующие тестирование и сборку.
	\item Make~--- инструмент для автоматизации процессов:
		сборки, тестирования, генерации отчетов.
		Упрощает управление зависимостями и воспроизводимость задач.
\end{enumerate}

\section{Описание логики работы ИС}

\subsection{Дерево функций}

\subsection{Диаграмма прецедентов}

\subsection{Диаграмма последовательности}

\section{Описание модели данных}

\section{Описание реализация ИС}

\subsection{Модуль предобработки}

\subsection{Модуль генерации отчетов}

\subsection{Пользовательский интерфейс}

Пример вызова программы показан в листинге~\ref{lst:ui:cli:drc:config}.

\begin{lstlisting}[
	language=Bash,
	caption={Пример вызова программы},
	label=lst:ui:cli:drc:config
]
python3 script.py --config path/to/config
\end{lstlisting}

Пример содержания конфигурационного файла продемонстрирован
в листинге~\ref{lst:cfg:yaml}.

\begin{lstlisting}[
	language=YAML,
	caption={Содержание YAML конфигурации},
	label=lst:cfg:yaml
]
stream:
	input: path/to/file.in.txt
	output: path/to/output/file.out.txt
	details: true
parsing:
	processes: 6
preprocessing:
	macros: {LAYER_M: 1, M_TP: "PRIMARY", ISNULL: null}
	variables: {VAR_MIN: 1, VAR_MAX: 234}
report:
	file: "path/to/report.md"
	mode: "markdown"
logging:
	level: DEBUG
\end{lstlisting}

\section{Функциональное тестирование}

\subsection{Модульное тестирование}

\subsection{Интеграционное тестирование}

\subsection{Тестирование пользовательского интерфейса}

\subsection{Нагрузочное тестирование}

\subsection{Верификация соответствия требованиям}

\section{Демонстрация корректности реализации}

\subsection{Примеры конвертации типовых операторов}

\subsection{Визуализация результатов}

\section{Описание работы пользователей с ИС}

