\chapter{Исследовательский раздел}

\section{Контекст разработки и характеристика целевой аудитории}

\subsection{Объект исследования}

\lipsum[1][1-6]

\subsection{Предметная область}

\lipsum[2-3]

\subsection{Целевые пользователи системы}

\lipsum[4-5]

\subsection{Рыночный и технологический контекст}

\lipsum[6-7]

\section{Область применения и функциональные границы ИС}

\section{Анализ существующих аналогов}

\subsection{Сравнительная таблица}

\subsection{Причины разработки собственного продукта}

\section{Формирование требований к ИС}

\subsection{Функциональные требования}

\subsection{Нефункциональные требования}

\subsection{Пользовательские требования}

\section{Анализ и управление рисками}

Матрица рисков~--- это инструмент управления проектом,
предназначенный для систематического выявления,
анализа и минимизации потенциальных угроз,
способных повлиять на достижение целей разработки.
Она позволяет заранее подготовиться к неблагоприятным сценариям,
сократив вероятность их возникновения и смягчив последствия.
В таблице~\ref{table:risk:matrix} представлена структура матрицы.

\begin{longtable}{|p{3.2cm}|p{3.5cm}|p{1.5cm}|p{2.5cm}|p{4cm}|}
	\caption{Матрица рисков} \label{table:risk:matrix} \\
	\hline
	\textbf{Название риска}
	& \textbf{Последствия}
	& \textbf{Кач. оценка риска}
	& \textbf{Стратегия реагирования}
	& \textbf{Мероприятия} \\ 
	\hline
	\endfirsthead
	\conttable{table:risk:matrix} \\
	\hline
	\textbf{Название риска}
	& \textbf{Последствия}
	& \textbf{Кач. оценка риска}
	& \textbf{Стратегия реагирования}
	& \textbf{Мероприятия} \\
	\hline
	\endhead
	Задержка в сроках разработки
		& Пропуск сроков сдачи проекта
		& Высокая
		& Смягчение
		& Регулярные встречи команды для мониторинга статуса задач. \\ \hline
	Нехватка ресурсов
		& Увеличение времени разработки, снижение качества
		& Низкая
		& Устранение
		& Определить критические ресурсы заранее
			и обеспечить их наличие. \\ \hline
	Ошибки в коде
		& Необходимость доработки, потенциальные сбои системы
		& Высокая
		& Превентивные меры
		& Регулярное рецензирование кода и тестирование. \\ \hline
	Неопределенность требований
		& Сложности с реализацией функций
		& Низкая
		& Гибкость
		& Обсуждение требований с заказчиком на всех этапах. \\ \hline
	Технические проблемы
		& Остановка работы, снижение производительности
		& Средняя
		& Резервное копирование
		& Настройка резервных копий и тестов на выявление проблем. \\ \hline
\end{longtable}

