\chapter{Экономический раздел}

Экономический раздел
выпускной квалификационной работы (ВКР)
охватывает планирование бюджета, оценку затрат
и расчет экономической эффективности внедрения разрабатываемого
ИТ-решения~--- программы-конвертора DRC
для открытых инструментов проектирования микросхем.

\section{Организация и планирование работ по теме}

\subsection{Определение участников}

В реализации ВКР задействованы следующие участники:

\begin{enumerate}
	\item Руководитель ВКР~---
		обеспечивает корректное формулирование задач,
		осуществляет мониторинг выполнения ключевых этапов,
		вносит требуемые изменения в процесс работы
		и проводит итоговую оценку результатов проекта.
	\item Консультант по экономической части~---
		координирует процесс организации и планирования работ,
		а также осуществляет анализ и расчёт финансовых затрат по проекту.
	\item Разработчик~---
		выполняет все этапы разработки,
		включая проведение тестовых испытаний готового решения
		и оформление проектной документации.
\end{enumerate}

На рисунке~\ref{fig:economic:interaction} показаны роли
и коммуникации между участниками проекта.

\begin{image}
	\begin{tikzpicture}[
		role/.style={rectangle, draw=black, align=center},
		arrow/.style={<->, >=triangle 45}
	]
		\node(student) [role]
			{Разработчик};
		\node(supervisor) [role, right=of student]
			{Руководитель};
		\node(consultant) [role, left=of student]
			{Консультант};

		\draw[arrow] (student) -- (supervisor);
		\draw[arrow] (student) -- (consultant);
	\end{tikzpicture}
	\caption{График взаимодействия участников}
	\label{fig:economic:interaction}
\end{image}

\subsection{Организация работ}

Организация работ осуществляется на базе календарного планирования,
которое включает в себя:

\begin{itemize}
	\item планирование содержания проекта и декомпозицию работ;
	\item определение последовательности работ;
	\item планирование сроков, длительностей и логических связей работ.
\end{itemize}

В таблице~\ref{table:ec:development:stages} приведён состав этапов
работы с участвующими в разработке исполнителями.

\begin{longtable}{|p{0.5cm}|p{4.5cm}|p{2.7cm}|p{3cm}|p{4cm}|}
	\caption{Этапы разработки}\label{table:ec:development:stages} \\ \hline
	\textbf{\No}
	& \textbf{Название этапа}
	& \textbf{Исполнитель}
	& \textbf{Трудоемкость, чел/дни}
	& \textbf{Продолжительность работ, дни} \\ \hline
	\endfirsthead
	\conttable{table:ec:development:stages} \\ \hline
	\endhead
	%\hline
	%\endfoot

	\multicolumn{4}{|r|}{\textbf{Итого}} & 68 \\ \hline
\end{longtable}

В таблице~\ref{table:ec:calendar:plan}
приведён типовой состав этапов проектирования и разработки ИТ-решения,
планирование сроков выполняется с учетом: шестидневной рабочей
(учебной) недели руководителя ВКР, консультанта по экономической части и
студента, выходных и праздничных дней, а также состава участвующих в разработке
исполнителей.
Первым днём планирования является первый день 8-го семестра.

\begin{longtable}{|p{5.8cm}|p{2cm}|p{2cm}|p{1.8cm}|p{3cm}|}
	\caption{Календарный план выполнения проекта}
	\label{table:ec:calendar:plan} \\ \hline
	\textbf{Этап}
	& \textbf{Дата начала}
	& \textbf{Дата окончания}
	& \textbf{Кол-во рабочих дней}
	& \textbf{Исполнители} \\ \hline
	\endfirsthead
	\conttable{table:ec:calendar:plan} \\ \hline
	\endhead
	%\multicolumn{6}{c}{}  % debug
	%\endlastfoot
	%\hline
	%\endfoot

\end{longtable}

\section{Расчет затрат на проведение работ}

\subsection{Сырье и материалы}

\subsection{Основная заработная плата}

\subsection{Дополнительная заработная плата}

\subsection{Страховые взносы}

\subsection{Амортизация}

\subsection{Прочие расходы}

\subsection{Полная себестоимость работ}

